%%%%%%%%%%%%%%%%%%%%%%%%%%%%%%%%%%%%%%%%%
% Beamer Presentation
% LaTeX Template
% Version 1.0 (10/11/12)
%
% This template has been downloaded from:
% http://www.LaTeXTemplates.com
%
% License:
% CC BY-NC-SA 3.0 (http://creativecommons.org/licenses/by-nc-sa/3.0/)
%
%%%%%%%%%%%%%%%%%%%%%%%%%%%%%%%%%%%%%%%%%

%----------------------------------------------------------------------------------------
%	PACKAGES AND THEMES
%----------------------------------------------------------------------------------------

\documentclass[8pt]{beamer}

\mode<presentation> {

% The Beamer class comes with a number of default slide themes
% which change the colors and layouts of slides. Below this is a list
% of all the themes, uncomment each in turn to see what they look like.

%\usetheme{default}
%\usetheme{AnnArbor}
%\usetheme{Antibes}
%\usetheme{Bergen}
\usetheme{Berkeley}
%\usetheme{Berlin}
%\usetheme{Boadilla}
%\usetheme{CambridgeUS}
%\usetheme{Copenhagen}
%\usetheme{Darmstadt}
%\usetheme{Dresden}
%\usetheme{Frankfurt}
%\usetheme{Goettingen}
%\usetheme{Hannover}
%\usetheme{Ilmenau}
%\usetheme{JuanLesPins}
%\usetheme{Luebeck}
%\usetheme{Madrid}
%\usetheme{Malmoe}
%\usetheme{Marburg}
%\usetheme{Montpellier}
%\usetheme{PaloAlto}
%\usetheme{Pittsburgh}
%\usetheme{Rochester}
%\usetheme{Singapore}
%\usetheme{Szeged}
%\usetheme{Warsaw}

% As well as themes, the Beamer class has a number of color themes
% for any slide theme. Uncomment each of these in turn to see how it
% changes the colors of your current slide theme.

%\usecolortheme{albatross}
%\usecolortheme{beaver}
%\usecolortheme{beetle}
%\usecolortheme{crane}
%\usecolortheme{dolphin}
%\usecolortheme{dove}
%\usecolortheme{fly}
%\usecolortheme{lily}
%\usecolortheme{orchid}
%\usecolortheme{rose}
%\usecolortheme{seagull}
%\usecolortheme{seahorse}
%\usecolortheme{whale}
%\usecolortheme{wolverine}

%\setbeamertemplate{footline} % To remove the footer line in all slides uncomment this line
\setbeamertemplate{footline}[page number] % To replace the footer line in all slides with a simple slide count uncomment this line

%\setbeamertemplate{navigation symbols}{} % To remove the navigation symbols from the bottom of all slides uncomment this line
}

\usepackage{graphicx} % Allows including images
\usepackage{booktabs} % Allows the use of \toprule, \midrule and \bottomrule in tables
\usepackage{tabularx}  % for 'tabularx' environment and 'X' column type
\usepackage{ragged2e}  % for '\RaggedRight' macro (allows hyphenation)
\newcolumntype{Y}{>{\RaggedRight\arraybackslash}X} 

\setcounter{figure}{0}

%----------------------------------------------------------------------------------------
%	TITLE PAGE
%----------------------------------------------------------------------------------------

\title[Gross Motion Planning]{Gross Motion Planning - Structure and Approach} % The short title appears at the bottom of every slide, the full title is only on the title page
\author{Bastian Lang} % Your name
\institute[BRSU] % Your institution as it will appear on the bottom of every slide, may be shorthand to save space
{
Master of Autonomous Systems \\ % Your institution for the title page
}
\date{April 6, 2015} 

\begin{document}


\listoffigures
\begin{frame}
\titlepage 
\end{frame}

%----------------------------------------------------------------------------------------
%	PRESENTATION SLIDES
%----------------------------------------------------------------------------------------

\begin{frame}
\frametitle{Overview}
General structure of the article
\begin{itemize}
\item Introduction
\item Part I - Nature of Motion Planning
\item Part II - Basic Issues and Steps in Motion Planning
\item Part III - Survey of Recent work
\item Part IV - Conclusions
\end{itemize}
\end{frame}

\begin{frame}
\frametitle{Part I - Nature of Motion Planning}
\begin{itemize}
\item Introduction
\item Part I - Nature of Motion Planning
\begin{itemize}
\item Definition of terms
\item Complexity of Motion Planning
\item Classification of Motion Planning Problems
\item Classification of Motion Planning Algorithms
\end{itemize}
\item Part II - Basic Issues and Steps in Motion Planning
\item Part III - Survey of Recent work
\item Part IV - Conclusions
\end{itemize}
\end{frame}

\begin{frame}
\frametitle{Purpose of Part I}
\begin{itemize}
\item Provide basic terminology
\item Give motivation
\item Describe common classification for problems and algorithms in MP
\end{itemize}
\end{frame}

\begin{frame}
\frametitle{Part II - Basic Issues and Steps in Motion Planning}
\begin{itemize}
\item Introduction
\item Part I - Nature of Motion Planning
\item Part II - Basic Issues and Steps in Motion Planning
\begin{itemize}
\item World Space vs Configuration Space
\item Object Sensing and Representation
\item Approaches to Motion Planning
\item Search Methods
\item Local Optimization of Motion
\end{itemize}
\item Part III - Survey of Recent work
\item Part IV - Conclusions
\end{itemize}
\end{frame}

\begin{frame}
\frametitle{Purpose of Part II}
\begin{itemize}
\item Show how MP formulations usually do look like
\item Give overview over existing approaches and methods in MP
\item Provide tools for understanding recent works in MP\\
(previous works and algorithms are often the basis of newer approaches)
\end{itemize}
\end{frame}

\begin{frame}
\frametitle{Part III - Survey of Recent Work}
\begin{itemize}
\item Introduction
\item Part I - Nature of Motion Planning
\item Part II - Basic Issues and Steps in Motion Planning
\item Part III - Survey of Recent work
\begin{itemize}
\item Stationary Environments
\item Time Varying Environments
\item Motion Planning with constraints
\item Movable Object Problem
\item Comparison Tables
\end{itemize}
\item Part IV - Conclusions
\end{itemize}
\end{frame}

\begin{frame}
\frametitle{Purpose and Structuring of Part III}
Purpose
\begin{itemize}
\item present recent work 
\item compare different contributions
\end{itemize}
Structuring
\begin{itemize}
\item Divided into Problem Classes
\item Further divided into different approaches
\end{itemize}
\end{frame}

\begin{frame}
\frametitle{Purpose of Part IV}
\begin{itemize}
\item Summarizes contribution to current state
\item Problems solved or closer to being solved
\item Still open problems
\end{itemize}
\end{frame}

\begin{frame}
\frametitle{Conclusion}
Very intuitive structuring...
\begin{itemize}
\item Part I allows to get the basics
\item Part II provides and explains the structuring of Part III
\item Part III shows the recent work and compares it
\item Part IV provides an overview over the current state
\end{itemize}
\end{frame}

\end{document} 
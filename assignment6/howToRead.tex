%%%%%%%%%%%%%%%%%%%%%%%%%%%%%%%%%%%%%%%%%
% Short Sectioned Assignment
% LaTeX Template
% Version 1.0 (5/5/12)
%
% This template has been downloaded from:
% http://www.LaTeXTemplates.com
%
% Original author:
% Frits Wenneker (http://www.howtotex.com)
%
% License:
% CC BY-NC-SA 3.0 (http://creativecommons.org/licenses/by-nc-sa/3.0/)
%
%%%%%%%%%%%%%%%%%%%%%%%%%%%%%%%%%%%%%%%%%

%----------------------------------------------------------------------------------------
%	PACKAGES AND OTHER DOCUMENT CONFIGURATIONS
%----------------------------------------------------------------------------------------

\documentclass[paper=a4, fontsize=11pt]{scrartcl} % A4 paper and 11pt font size

\usepackage[T1]{fontenc} % Use 8-bit encoding that has 256 glyphs
\usepackage{fourier} % Use the Adobe Utopia font for the document - comment this line to return to the LaTeX default
\usepackage[english]{babel} % English language/hyphenation
\usepackage{amsmath,amsfonts,amsthm} % Math packages

\usepackage{graphicx}
\usepackage{float}

\usepackage{sectsty} % Allows customizing section commands
\allsectionsfont{\normalfont\scshape} % Make all sections centered, the default font and small caps

\usepackage{fancyhdr} % Custom headers and footers
\pagestyle{fancyplain} % Makes all pages in the document conform to the custom headers and footers
\fancyhead{} % No page header - if you want one, create it in the same way as the footers below
\fancyfoot[L]{} % Empty left footer
\fancyfoot[C]{} % Empty center footer
\fancyfoot[R]{\thepage} % Page numbering for right footer
\renewcommand{\headrulewidth}{0pt} % Remove header underlines
\renewcommand{\footrulewidth}{0pt} % Remove footer underlines
\setlength{\headheight}{13.6pt} % Customize the height of the header

\numberwithin{equation}{section} % Number equations within sections (i.e. 1.1, 1.2, 2.1, 2.2 instead of 1, 2, 3, 4)
\numberwithin{figure}{section} % Number figures within sections (i.e. 1.1, 1.2, 2.1, 2.2 instead of 1, 2, 3, 4)
\numberwithin{table}{section} % Number tables within sections (i.e. 1.1, 1.2, 2.1, 2.2 instead of 1, 2, 3, 4)

\setlength\parindent{0pt} % Removes all indentation from paragraphs - comment this line for an assignment with lots of text

%----------------------------------------------------------------------------------------
%	TITLE SECTION
%----------------------------------------------------------------------------------------

\newcommand{\horrule}[1]{\rule{\linewidth}{#1}} % Create horizontal rule command with 1 argument of height

\title{	
\normalfont \normalsize 
\textsc{BRSU} \\ [25pt] % Your university, school and/or department name(s)
\horrule{0.5pt} \\[0.4cm] % Thin top horizontal rule
\huge Homework for Introduction to Scientific Working\\Assignment 6\\Summary % The assignment title
\horrule{2pt} \\[0.5cm] % Thick bottom horizontal rule
}

\author{Bastian Lang} % Your name

\date{\normalsize\today} % Today's date or a custom date

\begin{document}

\maketitle % Print the title
\newpage

\section{How to read a research paper}
The article "How to read a research paper" by Wes Huang deals with how to read a research paper.\\
In the beginning the author distinguishes the two kinds of papers the reader probably will read. Those are \textbf{conference papers} and \textbf{journal papers}. According to the author conference papers are rather short and have been published on a conference meeting, whereas journal papers are "more complete" papers that have been published in a journal.\\
The author then gives an overview over the main journals and conferences in robotics. For the searching of papers he gives some tools to use (like electronic library databases or citeseer and scholar.google). When reading a paper the reader always has to keep some critical questions in mind regarding the paper he is reading, what problems the authors are solving, if they are really addressing the problem, is there any new contribution through their work and so on.\\
In the end the author gives some advise on the order one should read a paper. He suggests starting with the title, the abstract and the introduction, get the overall structure of the paper, move on to the conclusion and depending on the level of detail one is interested continue with the other parts.

\section{How to read an engineering research paper}
The article "how to read an engineering research paper" by William G. Griswold handles the way to read and understand engineering research papers. After a motivation on why reading and understanding a paper can sometimes be hard and why a structured approach is needed, the author first explains how a typical research paper is structured. He then asks some questions that every reader should ask himself while reading a paper. Those questions are questions about the motivation of the authors, the proposed solution, the evaluation, the readers own analysis of the problem, the idea and the evaluation, about the contribution, the future directions of this research field, questions left and the take away message.\\
Afterwards the author of the paper advises the reader to take notes about and/or highlight certain parts of the paper and to also write down questions regarding the content.\\
In the last part of the paper the author suggests reading a paper always at least twice. Once for getting the big picture and the second time for trying to understand the details. 



\end{document}